\section{HISTORIA DE LOS PROCESOS DE DESARROLLO}
Uno de los grandes pasos dados en la industria del software fue aquel en que se
plasmó el denominado modelo en cascada. Dicho modelo sirvió como base para la
formulación del análisis estructurado, el cual fue uno de los precursores en este camino
hacia la aplicación de prácticas estandarizadas dentro de la ingeniería de software.
Propuesto por Winston Royce en un controvertido paper llamado "Managing the
Development of Large Software Systems" [Royce, 1970] este modelo intentaba
proponer una analogía de línea de ensamblaje de manufactura para el proceso de desarrollo de software, queriendo forzar predictibilidad en una entidad como el software
que como fue mencionado es algo complejo y que esta más relacionado con el
desarrollo de nuevos productos.