\begin{abstract}
	El desarrollo de software no es una tarea fácil. Prueba de ello es que existen numerosas propuestas metodológicas que inciden en distintas dimensiones del proceso de desarrollo. Por una parte tenemos aquellas propuestas más tradicionales que se centran especialmente en el control del proceso, documentación y planificación, estableciendo rigurosamente las actividades involucradas, los artefactos que se deben producir, y las herramientas y notaciones que se usarán. Estas propuestas han demostrado ser efectivas y necesarias en un gran número de proyectos, pero también han presentado problemas en otros muchos. Una posible mejora es incluir en los procesos de desarrollo más actividades, más artefactos y más restricciones, basándose en los puntos débiles detectados. Sin embargo, el resultado final sería un proceso de desarrollo más complejo que puede incluso limitar la propia habilidad del equipo para llevar a cabo el proyecto. Otra aproximación es centrarse en otras dimensiones, como por ejemplo el factor humano o el producto software. Esta es la filosofía de las metodologías ágiles, las cuales dan mayor valor al individuo, a la colaboración con el cliente y al desarrollo incremental del software con iteraciones muy cortas . Este enfoque está mostrando su efectividad en proyectos con requisitos muy cambiantes y cuando se exige reducir drásticamente los tiempos de desarrollo pero manteniendo una alta calidad. Las metodologías ágiles están revolucionando la manera de producir software, y a la vez generando un amplio debate entre sus seguidores y quienes por escepticismo o convencimiento no las ven como alternativa para las metodologías tradicionales. En este trabajo se presenta resumidamente el contexto en el que surgen las metodologías ágiles, sus valores, principios y comparación con las metodologías tradicionales actualmente utilizado por el área de Desarrollo de Sistemas del Ministerio de Hacienda. Entre otras cosas se describen brevemente las principales propuestas, especialmente Programación Extrema (eXtreme Programming, XP) la metodología ágil más popular en la actualidad.\\\\
	\textbf{Palabras clave:}  Procesos de Software, Desarrollo incremental, Metodologías Ágiles, eXtreme Programming.
		\end{abstract}