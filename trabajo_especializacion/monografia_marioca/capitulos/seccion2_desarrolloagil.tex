\section{\textbf{DESARROLLO ÁGIL}}
			En el 2001, Kent Beck y otros 16 notables desarrolladores, escritores y consultores (conocidos como la alianza ágil) firmaron el Manifiesto para el desarrollo ágil de software, el cual establecía:\vspace*{0.3in}
			Hemos descubierto mejores formas de desarrollar software al construirlo por nuestra cuenta y ayudar a otros a hacerlo. Por medio de este trabajo hemos llegado a valorar:
			\begin{itemize}
				\item A los individuos y sus interacciones sobre los procesos y las herramientas.
				\item Al software en funcionamiento sobre la documentación extensa.
				\item A la colaboración del cliente sobre la negociación del contrato.
				\item A la respuesta al cambio sobre el seguimiento de un plan.\\
			\end{itemize}
			Esto es, aunque los términos a la derecha tiene valor, nosotros valoramos más los aspectos de la izquierda.\vspace*{0.3in}
			Un manifiesto se asocia por lo general con un movimiento político emergente: aquel que ataca a la vieja vanguardia y sugiere un cambio revolucionario (en el mejor de los casos para mejorar).
			\subsection{¿Que es?}
			La ingeniería de software ágil combina una filosofía y un conjunto de directrices de desarrollo. La filosofía busca la satisfacción del cliente y la entrega temprana del software incremental; equipos de proyecto pequeños y con alta motivación; métodos informales; un mínimo de productos de trabajo de la ingeniería del software; y una simplicidad general del desarrollo. Las directrices de desarrollo resaltan la entrega sobre el análisis y el diseño (aunque estas actividades no se descargan), y la comunicación activa y continua entre los desarrolladores y los clientes.
			\subsection{¿Quien lo hace?}
			Los ingenieros de software y otros participantes del proyectos (gerentes, clientes y usuarios finales) trabajan juntos en un equipo ágil: un equipo con organización propia y que controla su propio destino. Un equipo ágil fomenta la comunicación y la colaboración entre todos los que trabajan en él.
			\subsection{¿Porque es importante?}
			El ambiente moderno de los negocios ocasiona que los sistemas basados en computadoras y los productos de software estén acelerados y en cambio continuo. La ingeniería del software ágil representa una opción razonable a la ingeniería convencional para ciertas clases de software y ciertos tipos de proyectos de software. Ha demostrado su utilidad al entregar sistemas exitosos con rapidez.
			\subsection{¿Cuales son los pasos?}
			El desarrollo ágil podría llamarse con mayor precisión "ingeniería del software ligera". Las actividades básicas del marco de trabajo - comunicación con el cliente, planeación, modelado, construcción, entrega y evolución - se conservan, pero estas se conforman como un conjunto mínimo de tareas que empuja al equipo de proyecto hacia la construcción y la entrega (habrán quienes argumenten que esto se hace a costa del análisis del problema y el diseño de la solución).
			\subsection{¿Cual es el producto obtenido?}
			Los clientes e ingenieros de software que han adoptado la filosofía ágil tienen la misma visión: el único producto de trabajo realmente importante es un "incremento de software" en funcionamiento, l cual se entrega al cliente en una fecha prometida.
			\subsection{¿Como puedo estar seguro de lo que he hecho correctamente.?}
			Si el equipo de software esta de acuerdo en que el proceso funciona y dicho equipo produce incrementos de software entregables que satisfacen al cliente.\footnote{Ingeniería del Software. Roger Pressman. Página 78, 79}
			\subsection{¿Que es la agilidad?}
			Agilidad se ha convertido en la palabra de moda en cuanto se describe un moderno proceso de software. Cualquiera es ágil. Un equipo ágil es un equipo rápido que responde de manera apropiada a los cambios. Estos son en gran parte, la materia del desarrollo de software. Cambios en el software que se va a construir, cambios entre los miembros del equipo, cambios debidos a las nuevas tecnologías, cambios de todo tipo qe pueden incidir en el producto que se construye o en el producto que crea el producto. En cualquier actividad de software se debe incluir un soporte para los cambios, esto es algo que adoptamos porque es el alma y el corazón del software. Un equipo ágil reconoce que el software lo desarrollan individuos que trabajan en equipo y que las actitudes de esta gente, y su capacidad para colaborar, son esenciales para el éxito del proyecto. De acuerdo con la visión de Jacobson, la insistencia en el cambio es el conductor primordial hacia la agilidad. Los ingenieros de software deben tener pies veloces si quieren ajustarse a los cambios rápido que describe Jacobson.\vspace*{0.3in}
			
			\definecolor{shadecolor}{rgb}{1,0.8,0.3}
			\begin{shaded}
				"La agilidad es dinámica, con contenido específico, ajustable al cambio de manera dinámica y orientada al crecimiento."\\
				\begin{flushright}
					\textbf{Steven Goldman et al.}
				\end{flushright}
			\end{shaded}
			La alianza ágil define 12 principios para quienes quieren alcanzar la agilidad:\\
			
			\begin{table}[htbp]
				\begin{center}
					\begin{tabular}{|r|l|}
						\hline
						Ítem & Principios \\
						\hline \hline
						1 & Nuestra mayor prioridad es satisfacer al cliente mediante la entrega temprana y continua de software valioso \\ \hline
						2 & Bienvenidos los requisitos cambiantes, incluso es fases tardías del desarrollo. La estructura de los procesos ágiles cambia para ventaja competitiva del cliente. \\ \hline
						3 & París \\ \hline
					\end{tabular}
					\caption{Principios de la alianza ágil.}
					\label{tabla:sencilla}
				\end{center}
			\end{table}